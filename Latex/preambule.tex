%%%%%%%%%% encodage et langue %%%%%%%%%%

\usepackage[utf8x]{inputenc}
\usepackage[T1]{fontenc}
\usepackage[english]{babel}
\usepackage[titletoc,title]{appendix}

\usepackage[usenames,dvipsnames,table]{xcolor}

\usepackage[hyphens]{url}
\usepackage{hyperref}
\usepackage{enumitem}

%\AfterPreamble{
\hypersetup{
 pdfauthor       = { },
 colorlinks      = true,
 linkcolor       = OliveGreen,
 linkbordercolor = OliveGreen,
 urlcolor 	    = Plum,
 citecolor		= Plum,
 urlbordercolor  = Plum,
 pdfstartview    = {XYZ = null null 1.00},
}
%}

%\usepackage[pdfstartview = {XYZ = null null 1.00},%
%colorlinks = true,
%linkcolor = OliveGreen,
%urlcolor = Plum,
%citecolor = Plum,
%%linkbordercolor = green,
%%urlbordercolor = magenta
%%allbordercolors = green %only works with colorlinks=false
%]{hyperref}


%%%%%%%%%% choix de la police %%%%%%%%%%

%\usepackage{lmodern} % la police classique
%\usepackage{mathpazo} % la belle police de univ-sba.dz
%\usepackage{mathptmx} % une très belle police
%\usepackage{fourier}
%\usepackage{beton,euler} % la police de math-france.fr	 %
\usepackage[bitstream-charter]{mathdesign}% la police des beaux cours d'autom'
%% MEFIANCE : la police 'mathdesign' redéfinit certains symboles du package 'amssymb'. Il faut exclure ledit paquet, dans ce cas. Il faut aussi charger 'xcolor' avant les paquets 'Tikz'.

%%%%%%%%%% mise en page %%%%%%%%%%

\usepackage[hmargin=1.5cm, vmargin=1.5cm]{geometry}
\usepackage{titling}

\usepackage{sectsty}
\sectionfont{\color{MidnightBlue}}
\subsectionfont{\color{RoyalBlue}}
\subsubsectionfont{\color{Periwinkle}}
\paragraphfont{\color{MidnightBlue}}


\usepackage{enumitem}
\usepackage{pdfpages}
%\usepackage{ulem} % retours à la ligne pour \underline{}

%\setlength{\parindent}{0cm}
\widowpenalty = 10000
\clubpenalty = 10000
%\pagenumbering{gobble}
% désactive la numération automatique des pages

\usepackage{lastpage}
%\makeatletter
%\renewcommand{\@evenfoot}%
%	{\hfil \upshape page {\thepage} de \pageref{LastPage}}
%\renewcommand{\@oddfoot}{\@evenfoot}
%\makeatother

%\AtEndDocument{\label{LastPage}}

\usepackage{fancyhdr}
\pagestyle{fancy}
\setlength{\headheight}{15pt}
\renewcommand{\headrulewidth}{1pt}
\renewcommand{\footrulewidth}{1pt}
\fancyhead[L]{}
\fancyhead[C]{}
\fancyhead[R]{\rightmark
}
\fancyfoot[L]{\theauthor}
\fancyfoot[C]{}
\fancyfoot[R]{\thepage/\pageref{LastPage}}


%%%%%%%%%% figures, images et graphiques %%%%%%%%%%

\usepackage{eso-pic}
\usepackage{graphicx}
\usepackage{array}
\usepackage{placeins}
\usepackage{wrapfig}
\usepackage[lofdepth,lotdepth]{subfig}
\usepackage{fancybox}
\usepackage{pgfplots}
\usepackage{transparent}


%\usepackage{tabularx}
% permet d'obtenir l'environnement "begin{tabularx}{<longueur en cm>}{<format colonnes> X (colonne de taille à adapater)}... \end{tabularx}

%\usepackage{colortbl}
% offre plus de possibilités pour le coloriage des tableaux

% pour ajuster l'espacement vertical et horizontal des tableaux
\renewcommand{\arraystretch}{1.3} % amplification de l'espacement vertical (pas une longueur!)
\setlength{\tabcolsep}{0.3cm} % espacement horizontal (inter-colonnes d'habituellement 6pt)

%%%%%%%%%% mathématiques, unités SI %%%%%%%%%%

\usepackage{amsmath}
\usepackage{mathtools,stmaryrd}
%\usepackage{amssymb}
\usepackage{textcomp}
\usepackage{siunitx}
\usepackage{dsfont}

%%%%%%%%%% insertion de codes informatiques %%%%%%%%%%

%\usepackage{listings}
%\definecolor{light-gray}{gray}{0.95}
%\lstdefinestyle{custom}{
%	belowcaptionskip=1\baselineskip,
%	breaklines=true,
%	inputencoding=latin1,
%	frame=L,
%	xleftmargin=\parindent,
%	language=Matlab,
%	showstringspaces=false,
%	basicstyle=\footnotesize\ttfamily,
%	keywordstyle=\bfseries\color{green!40!black},
%	commentstyle=\itshape\color{purple!40!black},
%	identifierstyle=\color{blue},
%	stringstyle=\color{orange},
%}

\usepackage{listingsutf8}
\definecolor{light-gray}{gray}{0.95}
\lstdefinestyle{custom_matlab}{
 belowcaptionskip=1\baselineskip,
 breaklines=true,
 inputencoding=latin1,
 frame=L,
 xleftmargin=\parindent,
 language=Matlab,
 showstringspaces=false,
 basicstyle=\footnotesize\ttfamily,
 keywordstyle=\bfseries\color{green!40!black},
 commentstyle=\itshape\color{purple!40!black},
 identifierstyle=\color{blue},
 stringstyle=\color{orange},
}

\usepackage{listingsutf8}
\definecolor{light-gray}{gray}{0.95}
\lstdefinestyle{custom_c}{
 belowcaptionskip=1\baselineskip,
 breaklines=true,
 inputencoding=latin1,
 frame=L,
 xleftmargin=\parindent,
 language=C,
 showstringspaces=false,
 basicstyle=\footnotesize\ttfamily,
 keywordstyle=\bfseries\color{green!40!black},
 commentstyle=\itshape\color{purple!40!black},
 identifierstyle=\color{blue},
 stringstyle=\color{orange},
}

%\usepackage{algorithm}
%\usepackage{algorithmic}
\usepackage[linesnumbered,lined,boxed,commentsnumbered,french]{algorithm2e}


%%%%%%%%%% commandes personnalisées %%%%%%%%%%

%\newenvironment{dessin}[1][1]
%{\shorthandoff{:!}\begin{center}\begin{tikzpicture}[scale=#1]}{\end{tikzpicture}\end{center}\shorthandoff{:!}}

\usepackage{xspace}
\newcommand{\itb}{\item[$\bullet$]}
\newcommand{\its}{\item[$\star$]}
%\newcommand{\cad}{c'est-à-dire\xspace}
\newcommand{\cad}{\textit{i.e.}\xspace}
\newcommand{\e}{{\rm e}}

%%%%%%%%%%

%\newcommand{\titre}[2]{
%\begin{titlepage}
%\begin{center}
%\begin{minipage}[t]{0.45\linewidth}
%\raggedright
%\includegraphics[height=0.35\textwidth]{#1}
%\end{minipage}
%\hfill
%\begin{minipage}[t]{0.45\linewidth}
%\raggedleft
%\includegraphics[height=0.35\textwidth,clip=true,trim = {90px 90px 90px 90px}]{#2}
%\end{minipage}\\
%\vspace{5cm}
%%%%%%%%
%\hrule
%\vspace{0.75cm}
%{\LARGE \bfseries \scshape %\sffamily
%M2 -- FESup INTRANET \\}
%\vspace{0.5cm}
%{\LARGE \bfseries \scshape %\sffamily
%\thetitle \\}
%\vspace{0.75cm}
%\hrule
%%%%%%%%
%\vspace{1cm}
%{\Large \theauthor %\sffamily
%\\}
%\vspace{2.5cm}
%\vfill
%%\includegraphics[width = 0.65\textwidth]{#3}
%%\vfill
%\end{center}
%\end{titlepage}
%}


%%%%%%%%%% Paquets capricieux

\usepackage{tikz,tkz-tab}
\usepackage[europeanresistors,cuteinductors,emptydiodes,siunitx]{circuitikz}
\usetikzlibrary{arrows,shapes,positioning,patterns,calc}
%\usepackage{schemabloc}
