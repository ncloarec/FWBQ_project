\section*{Notations}
We start with some notations we will use along this report.
\begin{itemize}[font= \color{blue} \large, label= $\bullet$]
 \item $p(\cdot)$ denotes the function $p: x \mapsto p(x)$.
\end{itemize}

\section*{Introduction}

The goal of this article \cite{FWBQ} is to compute efficiently the integrals of the form
$ \displaystyle \int _ { \X } f ( x ) p ( x ) \mathrm { d } x$
where $\X \subseteq \R ^ { d }$ is a measurable space,
$d \geq 1$ integer representing the dimension of the problem, $p$ a probability
density with respect to the Lebesgue measure on $\X$ and $f : \X \rightarrow \R$
 is a \textit{test}-function.

 We will use the common approximation
 \begin{equation}
\int _ { \mathcal { X } } f ( x ) p ( x ) \mathrm { d } x \approx \sum _ { i = 1 } ^ { n } w _ { i } f \left( x _ { i } \right)
 \end{equation}
 but of course the real challenge lies in the choice of sequences $\acc{x_i}$ and
 $\acc{w_i}$ :
  \begin{itemize}[font= \color{blue} \large, label= $\bullet$]
    \item \textbf{Monte Carlo}: $w_i = \frac{1}{n}$ and $x_i$ realization of multivariate
    random variable $X_i \stackrel{iid}{\sim} X$ where $X$ has $p(\cdot)$ as probability
    distribution.
    \item \textbf{Kernel herding}:
    \item \textbf{Quasi-Monte Carlo}:
  \end{itemize}

  In the \textbf{Frank-Wolfe Bayesian Quadrature}, we have
  \begin{itemize}[font= \color{blue} \large, label= \ding{43}]
\item $\acc{w_i}$ which appear naturally in the Bayesian Quadrature by taking the expectation of a posterior distribution  (described in section \ref{sec:BQ}),
\item $\acc{x_i}$ selected by the Frank-Wolfe algorithm in order to minimize a posterior variance (described in section \ref{sec:FW}).
  \end{itemize}

  The main interest of the method developed in \cite{FWBQ} is the super fast
  \textit{exponential} convergence to the true value of the integral compared to the other methods mentioned above.
  % use Bach presentation for convergence rates benchmark

\section{Theoretical framework}


\section{Bayesian Quadrature}
\label{sec:BQ}

\section{Frank-Wolfe algorithm}
\label{sec:FW}
